%%% Local Variables: 
%%% coding: utf-8
%%% mode: latex
%%% TeX-engine: xetex
%%% End: 

\documentclass[11pt, a4paper]{awesome-cv}

% Configure page margins with geometry
\geometry{left=1.4cm, top=.8cm, right=1.4cm, bottom=1.8cm, footskip=.5cm}

% Specify the location of the included fonts
\fontdir[fonts/]

% Color for highlights
% Awesome Colors: awesome-emerald, awesome-skyblue, awesome-red, awesome-pink, awesome-orange
%                 awesome-nephritis, awesome-concrete, awesome-darknight
\colorlet{awesome}{awesome-red}
% Uncomment if you would like to specify your own color
% \definecolor{awesome}{HTML}{CA63A8}

% Colors for text
% Uncomment if you would like to specify your own color
% \definecolor{darktext}{HTML}{414141}
% \definecolor{text}{HTML}{333333}
% \definecolor{graytext}{HTML}{5D5D5D}
% \definecolor{lighttext}{HTML}{999999}

% Set false if you don't want to highlight section with awesome color
\setbool{acvSectionColorHighlight}{true}

% If you would like to change the social information separator from a pipe (|) to something else
\renewcommand{\acvHeaderSocialSep}{\quad\textbar\quad}


% Available options: circle|rectangle,edge/noedge,left/right
% \photo{./examples/profile.png}
\name{Arthur}{Amalvy}
\position{Computer Science Engineer}
\address{La Pinède, Massiès Sud, Massiès et La Borie, 30124 Thoiras}
\mobile{+33620730174}
\email{arthur.amalvyoff@gmail.com}
% \homepage{www.posquit0.com}
\github{Aethor}
\gitlab{Aethor}
\linkedin{https://www.linkedin.com/in/arthur-amalvy-576030162/}
% \stackoverflow{SO-id}{SO-name}
% \twitter{@twit}
% \skype{skype-id}
% \reddit{reddit-id}
% \medium{madium-id}
% \googlescholar{googlescholar-id}{name-to-display}
%% \firstname and \lastname will be used
% \googlescholar{googlescholar-id}{}
% \extrainfo{extra informations}

%\quote{``Be the change that you want to see in the world."}

\begin{document}

% Print the header with above personal informations
% Give optional argument to change alignment(C: center, L: left, R: right)
\makecvheader

% Print the footer with 3 arguments(<left>, <center>, <right>)
% Leave any of these blank if they are not needed
\makecvfooter
  {\today}
  {Arthur Amalvy~~~·~~~Curriculum Vitae}
  {\thepage}


\cvsection{Experience}

\begin{cventries}

  \cventry
  {Master in Computer Science and Information Engineering} % Degree
  {National Central University} % Institution
  {Taiwan} % Location
  {September 2019 - July 2020 (current)} % Date(s)
  {
    \begin{cvitems} % Description(s) bullet points
      \item Dual-Degree program
      \item Courses at the university, focused on machine learning, deep learning and Natural Language Processing
      \item Research at the Intelligent Information Research Laboratory, specialised in Natural Language Processing
    \end{cvitems}
  }

  \cventry
  {Engineering Degree in Computer Science}
  {University of Technology of Belfort-Montbéliard}
  {Belfort, Territoire de Belfort, France}
  {September 2017 - July 2020 (current)}
  {
    \begin{cvitems}
      \item Specialisation in Image, Interaction and Virtual Reality (I2RV)
    \end{cvitems}
  }

  \cventry
  {Cloud Microservices Architect Intern}
  {Idemia}
  {Osny, Région Parisienne, France}
  {September 2018 - February 2019}
  {
    \begin{cvitems}
      \item Six-months engineering internship 
      \item Conception and creation of a cloud based biometric system
      \item Worked with several cloud-related technologies : AWS, Docker, Kubernetes, Flask, REST APIs, OpenAPI, Redis, RabbitMQ, Prometheus
    \end{cvitems}
  }

  \cventry
  {DEUTEC}
  {University of Technology of Belfort-Montbéliard}
  {Sévenans, Territoire de Belfort, France}
  {September 2015 - July 2017}
  {
    \begin{cvitems}
      \item Integrated prep at UTBM
    \end{cvitems}
  }

  \cventry
  {Baccalauréat}
  {Lycée Jacques Prévert}
  {Saint-Christol-Lès-Alès, Gard, France}
  {2015}
  {
    \begin{cvitems}
      \item Obtained with high honors ("mention bien")
    \end{cvitems}
  }

\end{cventries}


\cvsection{Skills}

\begin{cvskills}
  \cvskill{Languages}
  {French (Native speaker), English (C2 level), Chinese (Beginnner)}
  
  \cvskill{Programming Languages}
  {Python, Java, C (Basic skills in R, C++, Javascript / Typescript and others)}

  \cvskill{Machine Learning}
  {Classical machine learning, Neural Networks, Recurrent Neural Networks (LSTM / GRU), Transformers, Transfer-learning (BERT-based models...)}

  \cvskill{Main Tools}
  {PyTorch, huggingface transformers, nltk, spacy, scikit-learn, GTK+, Linux, Git, Docker, Kubernetes}
\end{cvskills}


\cvsection{Events}

\begin{cvhonors}
  \cvhonor
  {French Robotics Cup}
  {Participated in the French Robotics Cup twice in a row for UTBM}
  {La Roche sur Yon}
  {2018, 2019}

  \cvhonor
  {BnF Hackaton}
  {Participed in the 2018 "Bibliothèque nationale de France" Hackaton}
  {Paris}
  {2018}
\end{cvhonors}


\cvsection{Projects}

\begin{cventries}

  \cventry
  {\href{https://gitlab.com/Aethor/bert-quote-attribution}{\textbf{gitlab link}}}
  {Bert for Quote Attribution}
  {National Central University}
  {2020 (current)}
  {
    \begin{cvitems}
      \item Part of the master thesis at NCU
      \item An original method for automatic quote attribution in literary texts
      \item Use of deep spans representations
      \item First ever usage of a BERT-based model for automatic quote attribution
      \item System easily adaptable to other domains 
    \end{cvitems}
  }

  \cventry
  {\href{https://github.com/Aethor/nintent}{\textbf{github link}} | \href{https://drive.google.com/open?id=1UbUyf57dc8fzP_Z-UZKmebMoGD3Ei9tc}{\textbf{report link}}}
  {Nintent}
  {National Central University}
  {2019}
  {
    \begin{cvitems}
      \item Final project of National Central University Natural Language Processing class
      \item Proposal of a novel deep-learning based algorithm for semantic parsing
    \end{cvitems}
  }

  \cventry
  {\href{https://gitlab.com/Aethor/bert-quote-attribution}{\textbf{gitlab link}} | \href{https://drive.google.com/open?id=1yXpPkRviuXfG6A0qUQQn782hfrFmdzLI}{\textbf{report link}}}
  {Abstractcadabra project}
  {National Central University}
  {2019}
  {
    \begin{cvitems}
      \item Participation to the T-Brain AI Cup, a national Taiwan competition
      \item Usage of transfer-learning and a hierarchical model for sentences tagging
      \item The project obtained the 51st place of the leaderboard
    \end{cvitems}
  }

  \cventry
  {\href{https://gitlab.com/Aethor/deltaray}{\textbf{gitlab link}} | \href{https://drive.google.com/open?id=1Pek2899zQjD0A-Mq9wzFfWmRks-o3iNQ}{\textbf{report link}}}
  {DeltaRay}
  {University of Technology of Belfort-Montbéliard}
  {2019}
  {
    \begin{cvitems}
      \item Preliminary work for the master thesis at NCU
      \item Conversational character networks extraction in cinema screenplays
      \item 3D character networks visualisation prototype
    \end{cvitems}
  }

  \cventry
  {\href{https://drive.google.com/open?id=1q9ahp9iRNREYmjb1pVUp9wJLECDwfH35}{\textbf{report link (french)}}}
  {MorphoKit Online}
  {Idemia}
  {2018}
  {
    \begin{cvitems}
      \item Cloud-based version of an existing Idemia biometric development kit (SDK)
      \item Participation in all stages of the project : Design, testing strategy, security risks estimation, implementation, deployment...
    \end{cvitems}
  }

\end{cventries}


\end{document}
