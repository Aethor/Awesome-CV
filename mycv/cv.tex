%%% Local Variables: 
%%% coding: utf-8
%%% mode: latex
%%% TeX-engine: xetex
%%% End: 

\documentclass[12pt, a4paper]{awesome-cv}

% fr
\usepackage[francais]{babel}

% bibliography
\usepackage[sorting=none]{biblatex}
\addbibresource{biblio.bib}

% Configure page margins with geometry
\geometry{left=1.4cm, top=.8cm, right=1.4cm, bottom=1.8cm, footskip=.5cm}

% Specify the location of the included fonts
\fontdir[fonts/]

% Color for highlights
% Awesome Colors: awesome-emerald, awesome-skyblue, awesome-red, awesome-pink, awesome-orange
%                 awesome-nephritis, awesome-concrete, awesome-darknight
\colorlet{awesome}{awesome-red}
% Uncomment if you would like to specify your own color
% \definecolor{awesome}{HTML}{CA63A8}

% Colors for text
% Uncomment if you would like to specify your own color
% \definecolor{darktext}{HTML}{414141}
% \definecolor{text}{HTML}{333333}
% \definecolor{graytext}{HTML}{5D5D5D}
% \definecolor{lighttext}{HTML}{999999}

% Set false if you don't want to highlight section with awesome color
\setbool{acvSectionColorHighlight}{true}

% If you would like to change the social information separator from a pipe (|) to something else
\renewcommand{\acvHeaderSocialSep}{\quad\textbar\quad}


% Available options: circle|rectangle,edge/noedge,left/right
% \photo{./examples/profile.png}
\name{Arthur}{Amalvy}
\position{Doctorant en Traitement Automatique du Langage Naturel au Laboratoire Informatique d'Avignon}
\address{Avignon, France}
% \mobile{+33620730174}
\email{arthur.amalvy@univ-avignon.fr}
% \homepage{www.posquit0.com}
\github{Aethor}
\gitlab{Aethor}
% \linkedin{https://www.linkedin.com/in/arthur-amalvy-576030162/}
% \stackoverflow{SO-id}{SO-name}
% \twitter{@twit}
% \skype{skype-id}
% \reddit{reddit-id}
% \medium{madium-id}
\googlescholar{yHOkn78AAAAJ}{Arthur Amalvy}
%% \firstname and \lastname will be used
% \googlescholar{googlescholar-id}{}
% \extrainfo{extra informations}

%\quote{``Be the change that you want to see in the world."}

\begin{document}

% Print the header with above personal informations
% Give optional argument to change alignment(C: center, L: left, R: right)
\makecvheader

% Print the footer with 3 arguments(<left>, <center>, <right>)
% Leave any of these blank if they are not needed
\makecvfooter
  {\today}
  {Arthur Amalvy~~~·~~~Curriculum Vitae}
  {\thepage}



\cvsection{Formation}

\begin{cventries}

  \cventry
  {Doctorat en Informatique} % Degree
  {Laboratoire Informatique d'Avignon} % Institution
  {Avignon} % Location
  {Septembre 2021 - Courant} % Date(s)
  {
    \begin{cvitems} % Description(s) bullet points
      \item Titre de la thèse : \textit{Traitement du langage et modélisation de relations pour la représentation unifiée de documents narratifs}
      \item Encadré par Vincent Labatut et Richard Dufour
      \item Activité de recherche : publications d'articles scientifiques relatifs à la thèse, développement de logiciels permettant l'extraction automatique de réseaux de personnages
      \item Activité d'enseignement : C++, HTML/CSS, JavaFX, Android, Python
    \end{cvitems}
  }

  \cventry
  {Master en Informatique et ingénierie de l'information} % Degree
  {National Central University} % Institution
  {Taiwan} % Location
  {Septembre 2019 - Juillet 2020} % Date(s)
  {
    \begin{cvitems} % Description(s) bullet points
      \item Double-diplôme
      \item Suivi de plusieurs cours à l'université, notamment en apprentissage machine, apprentissage profond et traitement du langage naturel
      \item Classé respectivement 7e puis 4e sur plus de 110 étudiants du département informatique à l'issue des deux semestres
      \item Recherches à l'~\textit{Intelligent Information Research Laboratory (IISR)}, specialisé en traitement du langage naturel
      \item Mémoire de master sur l'extraction automatique et la visualisation de réseaux de personnages dans les romans
    \end{cvitems}
  }

  \cventry
  {Diplôme d'ingénieur en informatique}
  {Université de technologie de Belfort-Montbéliard}
  {Belfort, Territoire de Belfort}
  {Septembre 2017 - Juillet 2020}
  {
    \begin{cvitems}
      \item Spécialisation en image, interaction et réalité virtuelle (\textit{I2RV})
      \item Suivi de plusieurs cours relatifs à l'intelligence artificielle
      \item Premiers travaux sur l'extraction de réseaux de personnages sur des transcripts de cinéma
    \end{cvitems}
  }

  \cventry
  {DEUTEC}
  {Université de Technologie de Belfort-Montbéliard}
  {Sévenans, Territoire de Belfort}
  {Septembre 2015 - Juillet 2017}
  {
    \begin{cvitems}
      \item Prépa intégrée à l'UTBM
      \item Début de spécialisation en informatique
    \end{cvitems}
  }

  \cventry
  {Baccalauréat scientifique}
  {Lycée Jacques Prévert}
  {Saint-Christol-Lès-Alès, Gard}
  {2015}
  {
    \begin{cvitems}
      \item Obtenu avec mention bien
    \end{cvitems}
  }

  
\end{cventries}


\cvsection{Expérience professionnelle}

\begin{cventries}

  \cventry
  {Ingénieur en développement}
  {Prokov Editions}
  {Nancy, France}
  {Octobre 2020 - maintenant}
  {
    \begin{cvitems}
      \item Inclusion de fonctionnalités liées à l'intelligence artificielle dans un logiciel médical
      \item Analyse de données d'usage du logiciel
    \end{cvitems}
  }

  \cventry
  {Stage ingénieur}
  {Idemia}
  {Osny, Région Parisienne, France}
  {Septembre 2018 - Février 2019}
  {
    \begin{cvitems}
      \item Conception et implémentation d'un prototype service biométrique basé sur le cloud
    \end{cvitems}
  }

\end{cventries}


\nocite{*}
\printbibliography[title=Publications]

\end{document}
