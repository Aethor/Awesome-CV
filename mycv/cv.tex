%%% Local Variables: 
%%% coding: utf-8
%%% mode: latex
%%% TeX-engine: xetex
%%% End: 

\documentclass[11pt, a4paper]{awesome-cv}

% bibliography
\usepackage[sorting=none]{biblatex}
\addbibresource{biblio.bib}

% Configure page margins with geometry
\geometry{left=1.4cm, top=.8cm, right=1.4cm, bottom=1.8cm, footskip=.5cm}

% Specify the location of the included fonts
\fontdir[fonts/]

% Color for highlights
% Awesome Colors: awesome-emerald, awesome-skyblue, awesome-red, awesome-pink, awesome-orange
%                 awesome-nephritis, awesome-concrete, awesome-darknight
\colorlet{awesome}{awesome-emerald}
% Uncomment if you would like to specify your own color
% \definecolor{awesome}{HTML}{CA63A8}

% Colors for text
% Uncomment if you would like to specify your own color
% \definecolor{darktext}{HTML}{414141}
% \definecolor{text}{HTML}{333333}
% \definecolor{graytext}{HTML}{5D5D5D}
% \definecolor{lighttext}{HTML}{999999}

% Set false if you don't want to highlight section with awesome color
\setbool{acvSectionColorHighlight}{true}

% If you would like to change the social information separator from a pipe (|) to something else
\renewcommand{\acvHeaderSocialSep}{\quad\textbar\quad}


% Available options: circle|rectangle,edge/noedge,left/right
\photo[circle,left]{./mycv/id.jpg}
\name{Arthur}{Amalvy}
\position{Natural Language Processing Researcher}
\address{Avignon, France}
\mobile{+33620730174}
\email{arthur.amalvy@univ-avignon.fr}
% \homepage{www.posquit0.com}
\github{Aethor}
\gitlab{Aethor}
% \linkedin{https://www.linkedin.com/in/arthur-amalvy-576030162/}
% \stackoverflow{SO-id}{SO-name}
% \twitter{@twit}
% \skype{skype-id}
% \reddit{reddit-id}
% \medium{madium-id}
\googlescholar{yHOkn78AAAAJ}{Arthur Amalvy}
%% \firstname and \lastname will be used
% \googlescholar{googlescholar-id}{}
% \extrainfo{extra informations}

%\quote{``Be the change that you want to see in the world."}

\begin{document}

% Print the header with above personal informations
% Give optional argument to change alignment(C: center, L: left, R: right)
\makecvheader

% Print the footer with 3 arguments(<left>, <center>, <right>)
% Leave any of these blank if they are not needed
\makecvfooter
  {\today}
  {Arthur Amalvy~~~·~~~Curriculum Vitae}
  {\thepage}


\cvsection{Education}

\begin{cventries}

  \cventry
  {PhD in Natural Language Processing}
  {Avignon Computer Science Laboratory}
  {Avignon, France}
  {September 2021 - December 2024}
  {
    \begin{cvitems}
      \item Thesis title: \textit{Natural Language Processing for the Representation of Narrative Texts through Character Networks}
      \item Teacher assistant experience: C++, Advanced Algorithms, HTML/CSS, JavaFX, Android, Python, Innovation
    \end{cvitems}
  }

  \cventry
  {Master in Computer Science and Information Engineering} % Degree
  {National Central University} % Institution
  {Zhongli, Taiwan} % Location
  {September 2019 - July 2020} % Date(s)
  {
    \begin{cvitems} % Description(s) bullet points
      \item Dual-Degree program
      \item Courses at the university, focused on machine learning, deep learning and natural language processing
      \item Research at the Intelligent Information Research Laboratory (IISR), specialised in natural language processing
    \end{cvitems}
  }

  \cventry
  {Engineering Degree in Computer Science}
  {University of Technology of Belfort-Montbéliard}
  {Belfort, France}
  {September 2017 - July 2020}
  {
    \begin{cvitems}
      \item Specialisation in Image, Interaction and Virtual Reality (I2RV)
    \end{cvitems}
  }

  \cventry
  {DEUTEC}
  {University of Technology of Belfort-Montbéliard}
  {Sévenans, France}
  {September 2015 - July 2017}
  {
    \begin{cvitems}
      \item Integrated prep at UTBM
    \end{cvitems}
  }

  \cventry
  {Baccalauréat}
  {Lycée Jacques Prévert}
  {Saint-Christol-Lès-Alès, France}
  {2015}
  {
    \begin{cvitems}
      \item Obtained with high honors ("mention bien")
    \end{cvitems}
  }

\end{cventries}


\cvsection{Industry Experience}

\begin{cventries}

  \cventry
  {AI Engineer}
  {Prokov Editions}
  {Nancy, France}
  {October 2020 - September 2021}
  {
    \begin{cvitems}
      \item AI specialist for a medical software company
    \end{cvitems}
  }

  \cventry
  {Cloud Microservices Architect Intern}
  {Idemia}
  {Osny, France}
  {September 2018 - February 2019}
  {
    \begin{cvitems}
      \item Conception and creation of a cloud based biometric system
      \item Worked with several cloud-related technologies : AWS, Docker, Kubernetes, Flask, REST APIs, OpenAPI, Redis, RabbitMQ, Prometheus
    \end{cvitems}
  }

\end{cventries}


\cvsection{Skills}

\begin{cvskills}
  \cvskill{Languages}
  {French (Native speaker), English (C2 level), Chinese (Beginnner)}
  
  \cvskill{Programming Languages}
  {Mainly Python, C, Common Lisp}

  \cvskill{Machine Learning}
  {Classical machine learning, Neural Networks, RNN, Transformers, Transfer-learning (BERT-based models...)}

  \cvskill{Main Tools}
  {PyTorch, huggingface transformers, nltk, spaCy, scikit-learn, Linux, Git}
\end{cvskills}


\cvsection{Events}

\begin{cvhonors}
  \cvhonor
  {T-Brain AI-Cup}
  {Ranked 51st at the T-Brain AI Cup 2019}
  {Taiwan}
  {2019}
  
  \cvhonor
  {France Robotics Cup}
  {Participated to the France Robotics Cup 5 times}
  {La Roche sur Yon}
  {2018, 2019, 2022--2024}
\end{cvhonors}


% HACK: here to force a newpage
\newpage
\cvsection{Scientific Publications}

\nocite{*}
% \printbibliography[title=Scientific Publications]
\printbibliography[heading=none]


\cvsection{Open-source software}

\begin{cventries}

  \cventry
  {\href{https://github.com/CompNet/Renard}{\textbf{See on GitHub}}}
  {Renard (Relationship Extraction from NARrative Documents)}
  {}
  {2021-2024}
  {
    \begin{cvitems}
      \item A Modular character network extraction pipeline
    \end{cvitems}
  }

  \cventry
  {\href{https://github.com/CompNet/Tibert}{\textbf{See on GitHub}}}
  {Tibert}
  {}
  {2021-2024}
  {
    \begin{cvitems}
      \item A simple to use Python coreference resolution library
    \end{cvitems}
  }

  \cventry
  {\href{https://gitlab.com/sharpattack/blink}{\textbf{See on GitLab}}}
  {Blink}
  {}
  {2021-2024}
  {
    \begin{cvitems}
      \item A Python library to work with RPLidar A2M8
    \end{cvitems}
  }

\end{cventries}


\end{document}
