%%% Local Variables: 
%%% coding: utf-8
%%% mode: latex
%%% TeX-engine: xetex
%%% End: 

\documentclass[12pt, a4paper]{awesome-cv}

% Configure page margins with geometry
\geometry{left=1.4cm, top=.8cm, right=1.4cm, bottom=1.8cm, footskip=.5cm}

% Specify the location of the included fonts
\fontdir[fonts/]

% Color for highlights
% Awesome Colors: awesome-emerald, awesome-skyblue, awesome-red, awesome-pink, awesome-orange
%                 awesome-nephritis, awesome-concrete, awesome-darknight
\colorlet{awesome}{awesome-red}
% Uncomment if you would like to specify your own color
% \definecolor{awesome}{HTML}{CA63A8}

% Colors for text
% Uncomment if you would like to specify your own color
% \definecolor{darktext}{HTML}{414141}
% \definecolor{text}{HTML}{333333}
% \definecolor{graytext}{HTML}{5D5D5D}
% \definecolor{lighttext}{HTML}{999999}

% Set false if you don't want to highlight section with awesome color
\setbool{acvSectionColorHighlight}{true}

% If you would like to change the social information separator from a pipe (|) to something else
\renewcommand{\acvHeaderSocialSep}{\quad\textbar\quad}


% Available options: circle|rectangle,edge/noedge,left/right
% \photo{./examples/profile.png}
\name{Arthur}{Amalvy}
\position{Candidat à une thèse de doctorat}
\address{Nancy, France}
\mobile{+33620730174}
\email{arthur.amalvyoff@gmail.com}
% \homepage{www.posquit0.com}
\github{Aethor}
\gitlab{Aethor}
\linkedin{https://www.linkedin.com/in/arthur-amalvy-576030162/}
% \stackoverflow{SO-id}{SO-name}
% \twitter{@twit}
% \skype{skype-id}
% \reddit{reddit-id}
% \medium{madium-id}
% \googlescholar{googlescholar-id}{name-to-display}
%% \firstname and \lastname will be used
% \googlescholar{googlescholar-id}{}
% \extrainfo{extra informations}

%\quote{``Be the change that you want to see in the world."}

\begin{document}

% Print the header with above personal informations
% Give optional argument to change alignment(C: center, L: left, R: right)
\makecvheader

% Print the footer with 3 arguments(<left>, <center>, <right>)
% Leave any of these blank if they are not needed
\makecvfooter
  {\today}
  {Arthur Amalvy~~~·~~~Curriculum Vitae}
  {\thepage}



\cvsection{Formation}

\begin{cventries}

  \cventry
  {Master en Informatique et ingénierie de l'information} % Degree
  {National Central University} % Institution
  {Taiwan} % Location
  {Septembre 2019 - Juillet 2020} % Date(s)
  {
    \begin{cvitems} % Description(s) bullet points
      \item Double-diplôme
      \item Suivi de plusieurs cours à l'université, notamment :
        \begin{itemize}
          \item apprentissage machine et apprentissage profond 
          \item traitement du langage naturel
          \item traitement du langage naturel appliqué aux réseaux sociaux 
        \end{itemize}
      \item Classé respectivement 7e puis 4e sur plus de 110 étudiants du département informatique à l'issus des deux semestres
      \item Recherches à l'~\textit{Intelligent Information Research Laboratory (IISR)}, specialisé en traitement du langage naturel
      \item Mémoire de master sur l'extraction automatique et la visualisation de réseaux de personnages dans les romans
    \end{cvitems}
  }

  \cventry
  {Diplôme d'ingénieur en informatique}
  {Université de technologie de Belfort-Montbéliard}
  {Belfort, Territoire de Belfort}
  {Septembre 2017 - Juillet 2020}
  {
    \begin{cvitems}
      \item Spécialisation en image, interaction et réalité virtuelle (\textit{I2RV})
      \item Suivi de plusieurs cours relatifs à l'intelligence artificielle, dont :
        \begin{itemize}
          \item Optimisation et Recherche Opérationelle
          \item Intelligence Artificielle : Concepts Fondamentaux et Langages Dédiés
          \item Intelligence Artificielle pour Jeux Sérieux 
        \end{itemize}
      \item Premiers travaux sur l'extraction de réseaux de personnages sur des transcripts de cinéma
    \end{cvitems}
  }

  \cventry
  {DEUTEC}
  {Université de Technologie de Belfort-Montbéliard}
  {Sévenans, Territoire de Belfort, France}
  {Septembre 2015 - Juillet 2017}
  {
    \begin{cvitems}
      \item Prépa intégrée à l'UTBM
      \item Début de spécialisation en informatique :
        \begin{itemize}
          \item Programmation en C
          \item Algorithmes et structures de données
          \item Programmation orientée objet
        \end{itemize}
    \end{cvitems}
  }

  \cventry
  {Baccalauréat scientifique}
  {Lycée Jacques Prévert}
  {Saint-Christol-Lès-Alès, Gard, France}
  {2015}
  {
    \begin{cvitems}
      \item Obtenu avec mention bien
    \end{cvitems}
  }

  
\end{cventries}


\cvsection{Expérience professionnelle}

\begin{cventries}

  \cventry
  {Ingénieur en développement}
  {Prokov Editions}
  {Nancy, France}
  {Octobre 2020 - maintenant}
  {
    \begin{cvitems}
      \item Inclusion de fonctionnalités liées à l'intelligence artificielle dans un logiciel médical, telles que :
      \begin{itemize}
        \item Détection automatique de l'orientation d'un document
        \item Prédiction de la prochaine action utilisateur dans le but de fluidifier l'interface
      \end{itemize}
      \item Analyse de données d'usage du logiciel
    \end{cvitems}
  }

  \cventry
  {Stage ingénieur}
  {Idemia}
  {Osny, Région Parisienne, France}
  {Septembre 2018 - Février 2019}
  {
    \begin{cvitems}
      \item Conception et implémentation d'un prototype service biométrique basé sur le cloud
      \item Utilisation de plusieurs technologies liées au cloud
    \end{cvitems}
  }

\end{cventries}


\newpage

\cvsection{Compétences}

\begin{cvskills}
  \cvskill{Langages}
  {Anglais (Niveau C2), Chinois (débutant)}
  
  \cvskill{Langages de programmation}
  {Principalement Python, C et Java}

  \cvskill{Apprentissage machine}
  {Apprentissage ``classique'', réseaux de neurones, réseaux de neurones récurrents, transformers, apprentissage par transfert (modèles types \textit{BERT})}

  \cvskill{Outils}
  {PyTorch, huggingface transformers, nltk, spacy, scikit-learn}
\end{cvskills}


\cvsection{Évenements}

\begin{cvhonors}
  \cvhonor
  {T-Brain AI-Cup}
  {51e place à la T-Brain AI Cup 2019}
  {Taiwan}
  {2019}
  
  \cvhonor
  {Coupe de France de robotique}
  {Double participation à la coupe de france de robotique sous les couleurs de l'UTBM}
  {La Roche sur Yon}
  {2018, 2019}
\end{cvhonors}


\cvsection{Projets}

\begin{cventries}

  \cventry
  {Mémoire de master | \url{https://gitlab.com/Aethor/master-thesis}}
  {Natural Language Processing applied to Interactive Character Relationships Visualization in Novels}
  {National Central University}
  {2020}
  {
    \begin{cvitems}
      \item Mon mémoire de master, réalisé au NCU
      \item Utilisation de technique d'apprentissage prodond afin d'extraire des graphes de personnages de romans littéraires
    \end{cvitems}
  }

  \cventry
  {Attribution de la parole dans les romans | \url{https://gitlab.com/Aethor/bert-quote-attribution}}
  {Bert for Quote Attribution}
  {National Central University}
  {2020}
  {
    \begin{cvitems}
      \item La contribution centrale de ma thèse de master au NCU
      \item Une méthode originale d'attribution automatique de la parole dans les textes littéraires
      \item Utilisation d'un modèle basé sur \textit{BERT}
      \item Méthode facilement adaptable à d'autres problèmes
    \end{cvitems}
  }

  \cventry
  {Parsing sémantique d'instructions | \url{https://github.com/Aethor/nintent}}
  {Nintent}
  {National Central University}
  {2019}
  {
    \begin{cvitems}
      \item Projet final du cours de traitement du langage naturel du NCU
      \item Proposition d'un algorithme de parsing sémantique basé sur des méthodes de deep-learning
    \end{cvitems}
  }

  \cventry
  {T-Brain AI-Cup 2019 | \url{https://gitlab.com/Aethor/abstractcadabra-project}}
  {Abstractcadabra project}
  {National Central University}
  {2019}
  {
    \begin{cvitems}
      \item Utilisation d'apprentissage par transfert et d'un modèle hierarchique pour le tagging de phrases
      \item Le projet a obtenu la 51e place du classement, sur plus de 200 participants
      \item Performances de niveau top 10 après amélioration du système à posteriori
    \end{cvitems}
  }

  \cventry
  {BERT et reconnaissance d'entités nommées | \url{https://gitlab.com/Aethor/transformner}}
  {TransformNER}
  {National Central University}
  {2019}
  {
    \begin{cvitems}
      \item Projet effectué pendant le premier semestre au NCU
      \item Utilisation de BERT pour la reconnaissance d'entités nommées
      \item Utilisation de la librairie Hyperopt, permettant l'optimisation d'hyperparamètres
    \end{cvitems}
  }

  \cventry
  {Réseaux de personnages dans le cinéma | \url{https://gitlab.com/Aethor/deltaray}}
  {DeltaRay}
  {Université de Technologie de Belfort-Montbéliard}
  {2019}
  {
    \begin{cvitems}
      \item Travail préliminaire à mon mémoire de master au NCU
      \item Extraction de réseaux conversationnels de personnages dans des transcripts de cinéma
      \item Prototype permettant la visualisation de réseaux de personnages dynamiques en 3D
    \end{cvitems}
  }

  \cventry
  {Portage d'un SDK biométrique sur le cloud | \url{https://drive.google.com/open?id=1q9ahp9iRNREYmjb1pVUp9wJLECDwfH35}}
  {MorphoKit Online}
  {Idemia}
  {2018}
  {
    \begin{cvitems}
      \item Mon projet principal durant mon stage à \textit{Idemia}
      \item Portage d'un kit de développement biométrique existant sur le cloud
      \item Participation à toutes les étapes du projet : design, stratégie de tests, estimation des risques, implémentation, déploiement...
    \end{cvitems}
  }

\end{cventries}


\end{document}
